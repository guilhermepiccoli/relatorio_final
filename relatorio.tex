% Exemplo de relatório tecnico do IC (http://www.ic.unicamp.br/~reltech)

\documentclass[11pt,twoside]{article}
\usepackage{techrep-ic}

\usepackage[brazil]{babel}
\usepackage[utf8]{inputenc}

\usepackage
[pdftitle={Avaliacao do Impacto de Tecnicas de Formacao de Regioes no Gerenciamento de Code Cache},
pdfauthor={G. Vieira, A. Carvalho, G. Valente, G. Piccoli, J. de Lucca},
pdfsubject={Relatorio Final},
pdfkeywords={VMs, Region Formation, Code Cache, RAIn, x86},
pdfpagemode={UseNone},
bookmarks,
pdfstartview={FitH},
colorlinks,
linkcolor={black},
citecolor={black},
urlcolor={blue},
hyperindex]
{hyperref}


\begin{document}

%%% PÁGINA DE CAPA %%%%%%%%%%%%%%%%%%%%%%%%%%%%%%%%%%%%%%%%%%%%%%%
% 
% Número do relatório
\TRNumber{99}

% DATA DE PUBLICAÇÃO (PARA A CAPA)
%
\TRYear{13}  % Dois dígitos apenas
\TRMonth{12} % Numérico, 01-12

% LISTA DE AUTORES PARA CAPA (sem afiliações).
\TRAuthor{G. Vieira, A. Carvalho, G. Valente,\\G. Piccoli, J. de Lucca}

% TÍTULO PARA A CAPA (use \\ para forçar quebras de linha).
\TRTitle{Avaliação do Impacto de Técnicas de Formação\\ de Regiões no Gerenciamento de Code Cache}

\TRMakeCover

%%%%%%%%%%%%%%%%%%%%%%%%%%%%%%%%%%%%%%%%%%%%%%%%%%%%%%%%%%%%%%%%%%%%%%
% O que segue é apenas uma sugestão - sinta-se à vontade para
% usar seu formato predileto, desde que as margens tenham pelo
% menos 25mm nos quatro lados, e o tamanho do fonte seja pelo menos
% 11pt. Certifique-se também de que o título e lista de autores
% estão reproduzidos na íntegra na página 1, a primeira depois da
% página de capa.
%%%%%%%%%%%%%%%%%%%%%%%%%%%%%%%%%%%%%%%%%%%%%%%%%%%%%%%%%%%%%%%%%%%%%%

%%%%%%%%%%%%%%%%%%%%%%%%%%%%%%%%%%%%%%%%%%%%%%%%%%%%%%%%%%%%%%%%%%%%%%
% Nomes de autores ABREVIADOS e titulo ABREVIADO,
% para cabeçalhos em cada página.
%
\markboth{Vieira, Carvalho, Valente, Piccoli e de Lucca}{Teste}
\pagestyle{myheadings}

%%%%%%%%%%%%%%%%%%%%%%%%%%%%%%%%%%%%%%%%%%%%%%%%%%%%%%%%%%%%%%%%%%%%%%
% TÍTULO e NOMES DOS AUTORES, completos, para a página 1.
% Use "\\" para quebrar linhas, "\and" para separar autores.
%
\title{Avaliação do Impacto de Técnicas de Formação\\ de Regiões no Gerenciamento de Code Cache}

\author{ Gilvan Vieira\thanks{IC} \and
Alisson Linhares\footnotemark[1] \and Gilberto Valente\thanks{FEM} \and Guilherme Piccoli\footnotemark[1] \and Jonatas de Lucca\thanks{Eldorado} }

\date{2013-12-11} %13,12,11 kkkkk

\maketitle

%%%%%%%%%%%%%%%%%%%%%%%%%%%%%%%%%%%%%%%%%%%%%%%%%%%%%%%%%%%%%%%%%%%%%%

\begin{abstract} 
  Resumo % dois espaços antes do inicio da linha
  
\end{abstract}

\section{Introdução}
Introdução

\subsection{Motivação}
Motivação

\section{Objetivos}
Objetivos

\section{Trabalhos Relacionados}
Trabalhos Relacionados

\section{Metodologia}
Metodologia\cite{thesis-zinsly}

\subsection{algumas...}
tipo code cache, formação de regioes, metricas

\section{Resultados}
Resultados

\section{Conclusões}
Conclusões


% bibliografia
\bibliographystyle{ieeetr}
\bibliography{bibliografia}


\end{document}
